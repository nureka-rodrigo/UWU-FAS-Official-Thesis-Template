\section{Literature Review}
\label{sec:literature}

\noindent Organize this chapter to build a logical bridge from existing knowledge
to your proposed solution. Group sources by concept or methodology rather than
summarizing them chronologically. Aim to synthesise insights, highlight trends,
and call out unresolved tensions.

\subsection{Search Strategy and Scope}
Describe the databases, keywords, and inclusion/exclusion criteria used when
collecting sources. Mention the time span covered and justify why certain
literatures (e.g., interdisciplinary research) were or were not included. Use
tables or appendices to document the search protocol if it is extensive.

\subsection{Theoretical and Conceptual Foundations}
Discuss the theories, models, or frameworks that underpin the study. Explain
how each construct relates to your research objectives. If you adapt an
existing framework, articulate the modifications and provide a rationale.

\subsection{Empirical Studies and Critical Analysis}
Synthesize past empirical findings, focusing on methodologies, datasets, and
outcomes most relevant to your project. Contrast strengths and weaknesses
across studies—for instance, note how deep learning approaches have improved
detection accuracy but often rely on compute-heavy architectures
\citep{Rajadurai2020}. Make sure to connect the critique back to the research
gap identified in Chapter~\ref{sec:introduction} (update the label to match
your final structure).

\begin{table}[H]
    \centering
    \caption{Illustrative comparison matrix. Replace or extend the entries to
        summarize key studies or technologies relevant to your topic.}\label{tab:related-work}
    \begin{tabular}{p{3.5cm}p{3cm}p{3cm}p{3.5cm}}
        \hline
        Study                       & Dataset/Context     & Methodology         & Reported Contribution                          \\
        \hline
        Abdul lateef et al. (2019)  & UNSW-NB15           & Deep neural network & Improved accuracy but limited interpretability \\
        Rajadurai and Gandhi (2020) & Wireless IoT traces & Stacked ensemble    & Enhanced detection on imbalanced data          \\
        \hline
    \end{tabular}
\end{table}

\subsection{Summary and Identified Gap}
Conclude the chapter with a concise synthesis. Reiterate the main themes,
clarify the limitations of prior work, and state how your project addresses the
gap. This paragraph should smoothly transition into the methodology chapter by
signalling the research design choices you made in response to the identified
needs.
