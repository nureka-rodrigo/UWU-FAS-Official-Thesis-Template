\section{Results and Discussion}
\label{sec:results-discussion}

\noindent Present findings objectively before interpreting them. Align the
structure with the research objectives and hypotheses introduced earlier.

\subsection{Overview of Findings}
Begin with a short paragraph summarizing the key outcomes. Mention whether each
objective was achieved and direct readers to detailed subsections for evidence.

\subsection{Quantitative or Qualitative Results}
Organize results thematically. For quantitative work, report descriptive
statistics followed by inferential analyses. Include measures of variance and
effect sizes. For qualitative work, describe emergent themes with
representative quotes or artifacts.

\begin{table}[H]
    \centering
    \caption{Template for reporting evaluation metrics. Adjust rows and columns to
        match your experiment design.}\label{tab:evaluation}
    \begin{tabular}{p{4cm}p{2cm}p{2cm}p{2cm}}
        \hline
        Model/Variant              & Precision     & Recall        & F1-score      \\
        \hline
        Baseline (cite source)     & 0.78          & 0.74          & 0.76          \\
        \textbf{Proposed approach} & \textbf{0.89} & \textbf{0.87} & \textbf{0.88} \\
        Variant with ablation      & 0.84          & 0.83          & 0.83          \\
        \hline
    \end{tabular}
\end{table}

\noindent Reference all tables and figures in prose (e.g.,
``Table~\ref{tab:evaluation} compares the model variants across the primary
metrics''). Maintain consistency in decimal places and units across the chapter.

\subsection{Discussion}
Interpret the findings, linking back to literature reviewed in
Chapter~\ref{sec:literature}. Discuss why certain patterns emerged, how they
compare with prior work, and what implications they have for the research
problem. Use subsubsections if you need to separate discussions for each
research question.

\subsection{Threats to Validity}
Reflect on limitations such as sample size, instrument calibration, or external
factors. Explain how these limitations might influence the interpretation of
the results and suggest mitigations or follow-up studies.

\subsection{Chapter Summary}
Close with a concise recap that sets the stage for
Chapter~\ref{sec:conclusion}. Highlight the most important findings and remind
the reader how they address the research objectives.
