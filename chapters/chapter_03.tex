\section{Methodology}
\label{sec:methodology}

\noindent This chapter should enable another researcher to replicate or audit your
study. Switch to past tense for completed procedures and present tense when
describing the rationale behind methodological choices.

\subsection{Research Design}
State the overarching design (e.g., experimental, quasi-experimental, survey,
design science) and justify why it aligns with the research objectives. Include
a brief diagram or narrative that outlines the phases of the study.

\subsection{Participants and Data Sources}
Describe who or what was studied. Specify sampling techniques, inclusion and
exclusion criteria, and recruitment procedures. For secondary datasets, include
the provider, access date, and licensing terms. Summarize demographics in a
table if appropriate and refer to it in the text.

\subsection{Materials, Tools, and Instruments}
Document hardware specifications, software versions, questionnaires, or sensors
used. When using open-source libraries, cite them formally. If you created an
instrument, explain the validation process (pilot testing, reliability metrics,
expert review, etc.).

\subsection{Procedure and Workflow}
Detail the chronological steps executed during the study. Ensure every
objective from Chapter~\ref{sec:introduction} is addressed. When helpful, use
an enumerated list such as:

\begin{enumerate}
    \item Data acquisition and preprocessing.
    \item Model development and training, including hyper-parameter tuning criteria.
    \item Evaluation using predefined metrics and baseline comparisons.
    \item Documentation of findings for reporting and reproducibility.
\end{enumerate}

Include sequence diagrams or workflow charts when they clarify complex
processes.

\subsection{Data Analysis Techniques}
Explain statistical tests, qualitative coding schemes, or machine learning
algorithms employed. Identify the assumptions behind each technique and how
they were verified. Mention open-source scripts or notebooks stored in an
appendix or repository for transparency. Reference equations where necessary
using exttt{equation} environments.

\subsection{Validity, Reliability, and Ethical Considerations}
Discuss steps taken to ensure internal and external validity. Describe measures
to protect participant privacy, data security, and ethical compliance (e.g.,
Institutional Review Board approval reference number). If limitations exist in
the methodology, acknowledge them and explain mitigation strategies.

\begin{figure}[H]
    \centering
    \includegraphics[width=0.75\textwidth]{figures/Figure.png}
    \caption{Insert a process diagram here. Replace the placeholder image and update the narrative so it aligns
        with your actual methodology.}\label{fig:workflow}
\end{figure}

\noindent Conclude with a short paragraph that previews how the collected data
feed into the subsequent chapter on results and discussion.
