\section{Conclusions and Recommendations}
\label{sec:conclusion}

\noindent This chapter synthesizes the entire study. Avoid introducing new
results; instead, interpret what has already been presented and translate it into
actionable guidance.

\subsection{Summary of Findings}
Restate the research aim and objectives, then concisely recap how each
objective was addressed. Reference key tables or figures from
Chapter~\ref{sec:results-discussion} to remind readers where supporting
evidence can be found.

\subsection{Contributions}
List theoretical, methodological, and practical contributions. Be explicit
about who benefits (e.g., scholars, practitioners, policy makers). If outputs
such as datasets or software were produced, mention where they are hosted.

\subsection{Recommendations}
Translate findings into recommendations for stakeholders. Structure this
section using bullet points or numbered lists if multiple audiences are
targeted. Each recommendation should link back to the evidence that justifies
it.

\subsection{Limitations and Future Work}
Summarize major limitations acknowledged earlier and propose realistic
directions for future research. Mention how upcoming work could extend the
scope or overcome current constraints.

\subsection{Closing Remarks}
End with a reflective paragraph that underscores the broader significance of
the study and invites readers to adopt or build upon the presented work.
