\section{Introduction}
\label{sec:introduction}

\noindent This chapter establishes the context for the study and convinces the reader why
the research topic matters. Write in the past tense when describing prior work and
in the present tense when outlining the structure of the thesis. Keep paragraphs
focused: each should develop one idea that supports your overarching argument.

\subsection{Background and Motivation}
Provide a concise overview of the domain and explain the real-world problem
that motivated the project. Anchor the discussion with authoritative sources,
citing each claim using reference keys from \texttt{references.bib} (for
example, ``network-based detection continues to evolve due to increasing
traffic volume'' \citep{Abdullateef2019}). Clarify any technical concepts early
to ensure readers outside your specialization can follow the story.

\subsection{Problem Statement}
State the core research problem in one or two precise sentences. Detail the gap
that persists in the literature or practice and justify why addressing it is
timely. Avoid vague language such as ``improve system performance''—quantify
the deficiency whenever possible.

\subsection{Research Aim and Objectives}
Summarize the overall aim followed by 3--5 SMART objectives. Highlight
measurable outcomes (e.g., ``design and evaluate a deep learning classifier for
anomaly detection with accuracy above 95\% on benchmark datasets''). If
hypotheses are formulated, number them clearly for later reference.

\subsection{Scope and Limitations}
Define the boundaries of the work. Mention datasets, user groups, time frames,
or technologies deliberately excluded. Being explicit here prevents readers
from making unrealistic inferences about the results.

\subsection{Significance of the Study}
Explain who benefits from the research and how. A balanced paragraph should
touch on theoretical contributions (e.g., new models or insights) and practical
value for industry, the community, or policy.

\subsection{Notation and Abbreviations}
Summarize specialised terminology that recurs throughout the thesis. Before it
is used in the text, register each acronym in
\verb|preliminary_pages/abbreviations.tex| with the command
\verb|\newacronym{label}{ABBR}{Full form}|. Within the chapters, reference the
term using \verb|\gls{label}| on first mention so the glossary expands it to
the full phrase (e.g., ``\gls{AI}''). Subsequent mentions
can rely on \verb|\gls{label}| or the plural form \verb|\glspl{label}| as
needed. When a long form must be repeated for clarity, use
\verb|\acrfull{label}|. Remember to include the glossaries in the build
workflow by running \verb|makeglossaries main| after adding or updating
acronyms.

\subsection{Thesis Organization}
Close the chapter with a roadmap of the remaining chapters. Introduce each
chapter in one sentence and mention how it links back to the objectives.

\begin{figure}[H]
    \centering
    \includegraphics[width=0.7\textwidth]{figures/Figure.png}
    \caption{Example figure caption. Replace this placeholder with a diagram that
        contextualizes your research}\label{fig:overview}
\end{figure}

\noindent Figures should be crisp and legible when printed in grayscale. Use vector
graphics for diagrams whenever possible and keep captions self-explanatory. Refer
to each figure in the text immediately before it appears (for example,
``Figure~\ref{fig:overview} summarizes the proposed architecture'').

\begin{table}[H]
    \centering
    \caption{Illustrative table summarizing stakeholder needs or dataset
        properties. Tailor the columns to match your study.}\label{tab:motivating-factors}
    \begin{tabular}{p{4cm}p{8cm}}
        \hline
        Stakeholder        & Key Concern or Requirement                                            \\
        \hline
        Industry partner   & Desire for near real-time anomaly alerts with minimal false positives \\
        Academic community & Need for reproducible baselines to compare intrusion detection models \\
        End users          & Expect intuitive dashboards and clear incident explanations           \\
        \hline
    \end{tabular}
\end{table}

\noindent Refer to tables in the narrative (for example, ``Table~\ref{tab:motivating-factors}
summarizes the expectations that shaped the objectives''). Keep the narrative
flowing rather than repeating table content verbatim.
